\section{Conclusion and \\ Future Directions}
\label{sec:conclude}

In our exploration of low-depth quantum architectures, we have found
close connections with compilation,
error-correction procedures, parallel classical
control, approximative algorithms, uninitialized qubits, magic state
distillation, and Hamiltonian simulation.

%%%%%%%%%%%%%%%%%%%%%%%%%%%%%%%%%%%%%%%%%%%%%%%%%%%%%%%%%%%%%%%%%%%%%%%%%%%%%%
\subsection{Uncomputing Ancillae}

Now that depth optimizations have bottomed out in the constant-depth case,
we can turn to optimizing for constant-size circuits or even numerically
optimizing the other circuit resources, especially circuit space.
This optimization may take the form of Bennett's intermediate uncomputing
techniques for reversible circuits \cite{Bennett1973}.
%which have already been used to convert any out-of-place, possibly irreversible,
%function to an in-place, reversible one, possibly with garbage bits.
In
particular the 2D factoring architecture of Ref. \cite{Pham2012b} has
depth $O(\log^2 n)$ but width $O(n^4)$ in the worst case due to the garbage
bits produced by the constant-depth addition, giving a circuit space of
$O(n^4 \log^2 n)$. This is quite an onerous burden
to maintain througout the entire circuit, especially since the majority of
these garbage bits are produced at the beginning, close to the inputs.
Since the circuit has polylogarithmic depth, we could uncompute ancillae back
towards the inputs and then recompute them back to the ``edge'' of the current computation
state in several waves, each successively close to the output qubits
like the waves of the ocean on a beach during a rising tide, without increasing
the size beyond a $O(\log n)$ factor.
%The circuit size of the current
%implementation is $O(n^3)$, so there is some room for improvement.
%Although this may not lead to an asymptotic improvement in circuit space,
%the calculation of exact numerical resources in this new model may be of
%great practical importance in near-future experiments.

%%%%%%%%%%%%%%%%%%%%%%%%%%%%%%%%%%%%%%%%%%%%%%%%%%%%%%%%%%%%%%%%%%%%%%%%%%%%%%
\subsection{Modular Computation States}

On the note of large monolithic computation states, it may be preferrable
to compute a circuit in separate modules and keep them unentangled as long
as possible. By preparing multiple identical modules in parallel, if any
of them fail error detection and correction, it may be discarded and an
equivalent module chosen to take its place. This is reminiscent of
Knill's post-selection scheme \cite{Knill2004}.

This may be another reason to prefer the phase estimation procedure of
Ref. \cite{Kitaev2002} which does not use a QFT and enables one to gradually
build up entanglement of all qubits with each other.

%%%%%%%%%%%%%%%%%%%%%%%%%%%%%%%%%%%%%%%%%%%%%%%%%%%%%%%%%%%%%%%%%%%%%%%%%%%%%%
\subsection{A Quantitative Theory of Magic}

Another interesting direction to pursue is in the distillation of magic
states. 
Wim van Dam has proposed the following conjecture about the interconversion
between
magic states and non-magic states.

\begin{equation}
\ket{H}^{\otimes n} \rightarrow \rho^{\otimes (c_1 + o(1))n}
\end{equation}

\begin{equation}
\rho^{\otimes n} \rightarrow \ket{H}^{\otimes (c_2 + o(1))n}
\end{equation}

We would like to determine if this relationship is valid and if so,
what is the relationship between the constants $c_1$
and $c_2$ and for what parameters of $\rho$. This may take inspiration
from the interconversion of different entangled resources in the LOCC
setting \cite{Ambainis2002}.

%%%%%%%%%%%%%%%%%%%%%%%%%%%%%%%%%%%%%%%%%%%%%%%%%%%%%%%%%%%%%%%%%%%%%%%%%%%%%%
\subsection{Farewell and Acknowledgments}

Although many circuits of interest have already been reduced to constant-depth,
which is the lowest depth there is, we hope that the proposed directions above
provide further avenues for interesting work. The author wishes to thank his
committee for their consideration and advice regarding his general examination:
Aram Harrow, Steve Flammia, Mark Oskin,
Luis Ceze, Krysta Svore, and Boris Blinov. He also wishes to thank Johnny Yan
and David Rosenbaum in the quantum theory group at UW for useful discussions.

He was listening mostly to M83, Flux Pavilion, Beats Antique, Feist,
and the Decemberists while writing this report. The typeface
is Palatino, and the source is available at \url{https://github.com/ppham/generals}.