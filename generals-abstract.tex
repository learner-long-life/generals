\section{Abstract}

In this report, we consider techniques for constructing
quantum circuits in architectures with realistic
parameters: nearest-neighbor constraints, parallel classical control,
fault-tolerant encodings, uninitialized qubits, and approximative operation.
Low-depth quantum architecture is an emerging
theme in quantum computing research with both theoretical and
practical aspects. On the theoretical side, we can characterize
circuits by quantum complexity classes defined by their depth and determine
interesting relationships between the equivalence or non-equivalence
of certain basic gates. This in turn, allows us to compare these
relationships with those of the corresponding classical complexity classes
and learn something more about where the apparent computing advance of quantum
computing over classical computing comes from. On the experimental side,
we would like our circuits to be as low-depth and parallel as possible
to take advantage of fast classical control and to complete in realistic
timescales (e.g. the lifetime of an adult human). The low-depth trend has
bottomed out in many cases, resulting in constant-depth circuits
for core quantum computing operations: arithmetic, unbounded
fanout, teleportation, modular exponentiation, and the quantum Fourier transform.
