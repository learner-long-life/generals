\begin{abstract}
In this report, we examine low-depth quantum architectures as a unifying
theme for classical control, parallelism in circuit construction,
error-correction, magic state distillation, and Hamiltonian simulation.
These topics have great importance both theoretically (in discovering
the limits of these techniques) and experimentally (in the efficient
implementation of algorithms). We examine uninitialized qubits
and approximate
algorithms and analyze the optimization of the QFT over time.
Since the low-depth trend has
bottomed out in many cases, resulting in a raft of
recent constant-depth circuits, we propose
future directions for quantum architecture research related to
circuit space, uncomputation, modular coherent states, and magic state
interconversion.
\end{abstract}