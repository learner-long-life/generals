\section{Parallelization Techniques}
\label{sec:parallel}

Here we survey two powerful techniques behind a recent trend of work on
constant-depth circuits. The first is the constant-depth unbounded fan-out.
The second is the parallelization of
commuting gates and the so-called OR reduction of H{\o}yer and {\v S}palek
\cite{Hoyer2002}.

%%%%%%%%%%%%%%%%%%%%%%%%%%%%%%%%%%%%%%%%%%%%%%%%%%%%%%%%%%%%%%%%%%%%%%%%%%%%%%
\subsection{Unbounded Fanout}

In classical
circuits, the ability to copy bits is taken for granted. However,
in quantum circuits,
it is impossible to create an unentangled copy of a general quantum state, a
principle known as ``no-cloning'' \cite{Nielsen2000}. However,
it \emph{is} possible to create entangled copies by transversally applying
the CNOT gate to corresponding qubits in the source and destination register.
One can copy one source qubit $\ket{\psi} = \alpha \ket{0} + \beta \ket{1}$
to $n$ target qubits in this manner according to Equation \ref{eqn:copy}.

\begin{equation}
(\alpha \ket{0} + \beta \ket{1}) \otimes \ket{0}^{\otimes n} \rightarrow
\alpha \ket{0}^{\otimes (n+1)} + \beta \ket{1}^{\otimes (n+1)}
\end{equation}

This can be done in logarithic depth with a cascade of CNOTs as shown
in Figure \ref{fig:cnot-cascade}.

\begin{figure}
\label{fig:cnot-cascade}
\caption{A cascade of CNOTs implementing unbounded fan-out in logarithmic depth}
\end{figure}

Moore and Nilsson showed it was equivalent in power to the parity gate
\cite{Moore1998}. H{\o}yer and {\v S}palek have studied it in the
context of constant-depth circuits over a fixed basis ($\textsc{QNC}_f^0$)
to approximate AND, OR, MAJORITY, THRESHOLD$[t]$, and EXACT$[t]$ gates.

This has also been used in adder circuits by Takahashi and Tani
\cite{Takahashi2009} to achieve $O(\log^* n)$ depth.
These gates are not just of theoretical interest; some physical systems
such as ion traps admit the implementation of efficient multi-qubit gates,
following the work of M{\o}lmer and S{\o}rensen \cite{Sorensen1999}.
Recently unbounded fanout gates have also been implemented in 2D architectures using
constant-depth cat-state creation in $n$ ancillary qubits to
implement $n$-qubit controlled operations \cite{Rosenbaum2012} and
factoring in polylogarithmic-depth on a \textsc{2D NTC} architecture
\cite{Pham2012b}.
An $n$-qubit cat state is shown in Equation \ref{eqn:cat}; it is the
maximally entangled $n$-qubit state which is an equal superposition of all
$\ket{0}$'s and all $\ket{1}$'s.
However, unbounded fanout consumes the cat state,
in that it is not possible to uncompute the cat state after its use.

\begin{equation}
\label{eqn:cat}
\ket{\Phi_n} = \frac{1}{\sqrt{2}} \(\ket{0}^{\otimes n} + \ket{1}^{\otimes n}\)
\end{equation}

We describe how unbounded fanout can be used later to parallelize the
quantum Fourier transform in Section \ref{sec:qft}.

%%%%%%%%%%%%%%%%%%%%%%%%%%%%%%%%%%%%%%%%%%%%%%%%%%%%%%%%%%%%%%%%%%%%%%%%%%%%%%
\subsection{OR Reduction of Hoyer and Spalek}

Gate applied to different qubits can be applied at the same time. This is
the classical definition of parallelism

A key insight of Ref. \cite{Hoyer2002} is that any two commuting operations
can be performed as the same time, provided that (1) we have access to
unbounded fan-out and (2) that can efficiently transform
into the basis where they are both diagonal and that we have access.
This second condition is the case for many
functions needed in quantum algorithms, for example modular arithmetic
\cite{Moore1998}. This technique is exploited in their construction of
circuits for constant-depth rotation by Hamming weight and rotation by
Hamming value.

The essential concept between the OR reduction is to reduce computing the
OR of $n$ qubits to computing the OR of $\log n$ qubits in constant depth.
Once this has been done, one can apply the reduction recursively to reduce
the OR of $\log n$ qubits to the OR of $\log \log n$ qubits, until we
are left with a constant number of qubits that we can take the OR of
in the obvious way. In this manner, H{\o}yer and {\v S}palek get a circuit
of $O\log^* n}$ depth, where $\log^*n$ is the iterated logarithm function.

This technique was later extended by Takahashi and Tani to create an exact circuit
for OR in constant-depth. This resolves the open question of the power of
various constant-depth circuit complexity classes, namely.

\begin{equation}
\label{eqn:qcc}
\textsc{QNC}_f^0 = \textsc{QAC}_f^0 = \textsc{QTC}_f^0
\end{equation}

Equation \ref{eqn:qcc} is surprising because there is a separation between
the classical equivalents. Namely, 

\Ipe{pict/rotation-hamming.ipe}
