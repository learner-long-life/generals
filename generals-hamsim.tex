\section{Hamiltonian Simulation}

The original proposal for quantum computing of Feynman in 1982 \cite{Feynman1982}
took the form of simulation. He pointed out that one problem that
quantum physical systems would be inherently better at than classical systems
would be simulating other quantum systems. Today, this is still an interesting
problem for the field. While Shor's factoring algorithm, and other algorithms
which are inspired by processing digital information, exhibit great speedups,
they are also far out of reach of today's experimental capabilities. To
factor a 4096-bit key in the most conservative implementation, with only
one layer of error-correction, would still take tens of thousands of qubits.
In contrast, the problem of simulating a quantum Hamiltonian of 40 particles
is within the reach of experiments within the next decade and already
far surpasses the abilities of all but the most powerful classical supercomputers
today.

Simulation itself is not exactly the same as computation. It
plays a complementary role to digital computing, much like the
now-defunct field of analog computing and other analog simulations. Instead
of digitizing variables and representing them indirectly in binary
encodings, abstracting away all physical effects, simulations encode variables
in other analog variables, where the physical state of the analog model
corresponds in some human-readable way to the simulated state of the
original variable.

Simulation Hamiltonians also has theoretical interest in that there are some
quantum physical systems which we can describe in a model
but can't even simulate quantum mechanical [need citation]. In that case,
there is an interesting twist in our expectations for a mathematical
to be \emph{predictive}. Furthermore, due to the "no fast-forwarding" theorem,
it turns out that even in a quantum simulation, we can't speed up time:
simulating a many-body quantum system for time $t$ still takes time
super-linear in $t$, although we can make it arbitrarily close to
linear \cite{Berry2005}.

Therefore, Hamiltonian simulation remains an interesting problem from the
time of Feynman until today, both for its practical applications, its
experimental feasibility, and its theoretical implications.

\subsection{Hamiltonians for Computer Scientists}

But what, you may be thinking to yourself, is a Hamiltonian?
In a quantum mechanical system, there are quantities which can be observed
by humans which are represented by Hermitian matrices. A Hermitian matrix
$H$ is self-adjoint, that is, $H=H^\dagger$, where $H^\dagger$ is the
conjugate transpose of $H$. Specifically, let $H_{r,c}$ denote the matrix
entry of $H$ in row $r$ and column $c$. Then the following holds:

\begin{equation}
H_{r,c} = H^*_{c,r}
\end{equation}

\subsection{Problem}

To state the problem more formally, suppose 