\section{Quantum Fourier Transform}

As an illustration of this new circuit model, with the resources of
\emph{circuit space} and \emph{unitialized ancillae} that we established in
the previous section, we will examine a core building block of many
quantum algorithms: the quantum Fourier transform (QFT).

[Presumably in the previous section, I discussed the limits of the depth-only
model, especially if it blows up in the number of ancillae, which is why I
introduced the idea of circuit space to begin with. Uninitialized qubits is
more of a curiosity at this point, and our only link to magic state distillation].

The QFT is responsible for most of our examples of an exponential quantum
speedup over classical algorithms (cite Shor's original paper etc), usually
by reducing problems to the framework of the hidden subgroup problem
(cite Chris Lomont's review paper). The most famous example is, of course,
Shor's algorithm for the prime factorization of large integers, which would
break the widely-deployed RSA cryptosystem. Therefore, it is a useful
module to study and optimize, and indeed, much attention has been devoted
to reducing its depth over the years, making it a great place to first apply
our model.

\subsection{Analogy to Classical Fourier Transform}

First, we will say a little bit more about the QFT itself. As the name implies,
it is inspired by the classical Fourier transform, which is used to convert
information in one representation (say, sound amplitudes over time) into another
representation (say, signal power over frequency). These representations are
equivalent, but some tasks are easier to do in one representation than another.
For example, in analyzing a sound recording in the time domain, many frequencies overlap and it
may be difficult to detect and remove a source of noise, let's say a low-frequency
hum associated with electrical mains power frequency. However, in the frequency
domain, it is easy to pick out a peak at 60 MHz, which is the frequency at
which North American power companies deliver power at alternating current.

\begin{figure}
\caption{Sound recording in the time and frequency domains}
\end{figure}

Does this equation have anything to do with the discrete classical Fourier
transform? I hope so. Fact check this!

\begin{equation}
x_j = \sum_{k=0}^{2^n-1} e^{-i2\pi j}y_k
\end{equation}

By analogy, in many quantum algorithms we can encode useful information
into the phases associated with the probability amplitudes of different
computational basis states. Our usual goal is to interfere these phases
both constructively and destructively, to maximize the probability of measuring
the computational basis state that corresponds to a correct answer to our
problem. The restriction is that we can only do this using local operations,
that, by operating on a few neighboring qubits at a time. Through this means,
we are effectively operating on many computational basis states in parallel,
due to the way quantum states combine through the tensor product structure.

[somewhere in the previous section, i unveil my famous spectrum picture
of quantum states, and demonstrate some examples of highly-entangled
states and product states]

\subsection{Product Representation and Phase Wheel Picture}

In one domain, we have information encoded in the bits of the computational
basis states, but these are just encodings, and you can imagine them
evolving happily in their own separate alternate universes until one of them
is selected by random chance by a measurement at the end.
In the second domain, we have the phases of each computational basis state,
and the way we can get different computational basis states to interact, or
to create more or less of them, is through two-qubit entangling gates.
Actually, now we are on thin ice, and I should think carefully about what I
want to say here, especially as it relates to universality.

Now the quantum analogues of the classical FT equations above.

\begin{equation}
\ket{x_j} = \sum_{k=0}^{2^n} e^{i\pi k / 2^n}\ket{y_k}
\end{equation}

What is the relationship between computational basis states and their
corresponding Fourier transform? I'm glad you asked that! Now I can break out
my famous rotating phase wheel picture.

\begin{figure}
\caption{My famous rotating phase wheel picture}
\end{figure}

The way to convert in between these two domains is the QFT, specifically
through the famous product representation. In particular, because of the
symmetry of the Fourier transform state, we can decompose the phase into
a controlled-rotation on each bit, where $j$ is decomposed into the
usual binary representation.

\begin{equation}
j = \sum_{\ell=0}^n 2^\ell j_\ell 
\end{equation}

\begin{equation}
\ket{x_j} = \otimes_{k=0}^{n} \frac{\ket{0} + e^{2\pi j_0 \cdots j_k}\ket{1}{\sqrt{2}}
\end{equation}

In particular, we can rotate the phase on the $\ket{1}$ part of each qubit by
increasingly more digits of significance of the overall phase shift $j$.
Somewhere I should also talk about phase significance, and how some qubits
have a greater influence on the overall phase of their corresponding
computational basis state. This depends, of course, on there first being
a $\ket{1}$ part to rotate. Starting from an initial state of $\ket{0}$, we
need to introduce a Hadamard to get an equal mix. Also, the rotation part is
achieved by controlled rotation gates $R_k$ of the following form:

\begin{equation}
R_k = \begin{array}{cc}
1 & 0 \\
0 & e^{i\pi / 2^k}
\end{array}
\end{equation}

Oh we should also note somewhere that the QFT can be generalized into the
Walsh-Hadamard transform, or that the Hadamard is just a QFT on 1-qubit.
Also, these QFT's all work on a modulus which is a power-of-two, $q=2^n$. This
is not always that useful in factoring, for example, since the modulus $q$ is
the number we are trying to factor, and if we already know it's a power-of-two,
well, you can see what I am getting at here.

\subsection{The Basic QFT Circuit in Depth $O(n^2)$}

Now at last, we are ready to see the circuit for the QFT. The most well-known
and straightforward representation is shown here from Nielsen & Chuang
\cite{Nielsen2000}, in Figure \ref{fig:qft-circuit}.

\begin{figure}
\label{fig:qft-circuit}
\caption{The QFT circuit from Nielsen & Chuang \cite{Nielsen2000}.
\end{figure}

Note that we are still in the old model of caring about depth, so this is
in the title of all the remaining subsections. But really, my goal is to show
you that depth isn't the whole picture, and using this nifty new concept of
circuit space, you get a much better view of whether you are getting anywhere
or not. Or my name isn't Peter Shor.

This circuit has both size and depth $O(n^2)$ for a QFT of size. It also
has the problem that is operates on qubits which may be very far apart, and
is not very parallel. I assume by this point that I have covered a section
on parallelism techniques, such as those employed by Hoyer and Spalek, and that
furthermore I figured out how to do those weird Nordic characters in the names of
Hoyer and Spalek. However, it doesn't use any ancillae (you know what those
are right, from the preliminary section on circuit resources).

The circuit space of this circuit is something something. To be calculated!

\subsection{The Nearest-Neighbor QFT Circuit in Depth $O(n)$}

To reduce this to nearest-neighbor architecture and also to operate on
disjoint sets of qubits at the same time, we use the following
construction by Fowler, Devitt, and Hollenberg \cite{Fowler2006}.

\subsection{The Semi-Classical QFT Circuit in Width}

This is the Beauregard circuit, but it is really for phase estimation.
I need to think some more before I go around indiscriminately trying to compare
it directly to other QFT circuits.


In its original form, and the one most well-known to beginning students of
quantum computing.