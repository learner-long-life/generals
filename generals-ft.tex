\section{Fault-Tolerance}
\label{sec:ft}

Now we turn to the subject of fault-tolerance, one of the largest and most
important areas of quantum computing. Since noise is the principal engineering
difficulty in realizing a functioning quantum computer, it is natural to
ask if it is possible, even in theory, to perform robust quantum computations
in the presence of errors.

Suppose that a single physical qubit has an error rate in the laboratory of
$p$. That is, for a given unit of time (this is scaled arbitrarily),
there is a probability $p$ that the
qubit will experience a bit flip or a phase flip (modeled as an unintended
$X$ or $Z$ gate), and with the remaining probability $1-p$ it stays the same.
More generally, we want to perform intentional gates on a qubit, and every gate
we perform (assuming they all take a time step) is a potential point for an
error to occur. If we have a quantum circuit with $S$ gates, and the errors
are independent but identically distributed, and therefore accumulate
linearly across the entire circuit. As is usually the case, we want our circuit
to have some total error probability equal to some small constant, say
$\epsilon = 1/4$. That way, we can repeat our circuit $4$ times, take the majority vote
of the answers, and on average we will have a high probability of getting the
correct answer from our circuit. What then, is our desired per-gate error $p'$
have to be? It is simply scaled by the size of the circuit, so
$p' < \frac{\epsilon}{S}$. In general, $p' < p$, in that our desired
error rate is less than (usually by several orders of magnitude) than our
actual error rate. How do we drive down the error rate?

To gain some perspective from classical computing, we can choose to protect
our quantum information by encoding it cleverly somehow, so that
errors that occur to the physical qubits (up to a certain point) do not
affect the logical qubit. In general, we denote by a $[[n,k,d]]$ code one that
encodes $k$ logical qubits (usually 1) into $n$ physical qubits with a distance of
$d$, that is, the code words differ from each other in $d$ qubits. In that
setting, we can correct up to $r = \lfloor d/2 \rfloor$ physical qubit errors.
In the simplest codes, the minimum we can do is
protect against a single-qubit error $d=3$. That is, in our encoding, we can detect
a single-qubit error when it happens. The Steane code, one of the earliest
quantum codes, is classified as $[[7,1,3]]$, meaning that $1$ logical qubit
is encoded into $7$ physical qubits, and single-qubit errors can be detected
and corrected.

Within an encoded logical gate operation, let's say there are at most $c$
places where a single-qubit error can occur. Since these errors occur
independently, by the same argument we used above for an entire circuit,
the total single-qubit error probability of the encoded gate is $cp$.

Now, we have a different error probability, one that an undetectable error
has occurred to our logical (encoded) qubit. The most likely such event is that
two errors occurred. Remember that our base, single-qubit error rate was
$p$, and for a given gate let's say there are $c$ different combinations of
places for two errors to occur. The event of failure is now $cp^2$, since the
event that two qubits fail simultaneously is independent. For the Steane
code, $c=10^4$ \cite{Nielsen2000}.

How can we drive down our error probability even farther?
We can \emph{concatenate} error correcting-codes. This means
that every physical qubit in our original encoding is now actually an
encoded qubit in an additional, recursive layer of encoding. If we think of
the base physical layer of qubits as Level 0, we can call the
initial level of encoding Level 1, and then this second level of encoding
as Level 2.

%\begin{figure}
%\caption{Insert figure of concatenated error correction}
%\end{figure}

What is the error probability in an encoded gate for the Level 2 block?
Treating Level 1 qubits as basic units, there are still $c$ locations for
pairs of errors to occur, but now the base error probability of gates for
Level 1 is $cp^2$. So the error probability for a Level 2 gate is
$c(cp^2)^2$.

In general, we can concatenate error-correcting codes up to $k$ levels, and
the per-gate error probability of a Level $k$ block is $p_k = c^{2^k-1}p^{2^k}$.
We've seen that $c$ can be quite large, so we need $p$ to be small enough
so that this value decreases with increasing $k$. In order for this scheme
to work

\begin{displaymath}
c(cp^2)^2 < 1
\end{displaymath}

which means that the ... well I want to say something about how $p$ has
to be initially smaller than $c$, or this whole scheme won't work.
$c^3 p^4 < 1$, $p < \sqrt[4/3]{1/c}$.

\subsection{Threshold Theorem}

One of the most celebrated results of quantum computing is the threshold
theorem, which states that if the individual error rates for qubit storage
over time and operations are low enough (beneath a particular threshold),
then we can use multiple levels of concatenation to drive down the errors
arbitrarily low, for example, to execute circuits of increasing size.

\begin{displaymath}
c^{2^k-1}p^{2^k} < p_th
\end{displaymath}

For the Steane code, depending on the particular architecture studied and
other parameters, $p_th \approx 10^{-5} - 10^{-6}$.
Error correction consumes vast resources. For the Steane code alone, without
taking into account nearest-neighbor or other architectural constraints, we
would require thousands or even tens of thousands of qubits just to execute a
circuit of a hundred gates. Therefore, people have looked for an alternative
error-correcting scheme, one that has a higher threshold.

%%%%%%%%%%%%%%%%%%%%%%%%%%%%%%%%%%%%%%%%%%%%%%%%%%%%%%%%%%%%%%%%%%%%%%%%%%%%%%
\subsection{Stabilizer Formalism and Surface Code}

Many error correcting codes can be studied in the \emph{stabilizer formalism},
a very useful framework developed by Daniel Gottesman in his thesis
\cite{Gottesman1997}. In this framework, the space of all codewords is
defined as the simultaneous $+1$ eigenstates of a set of Pauli group
operators, called the \emph{stabilizers} for the code. Often, it is common
to just work with the generating set for the stabilizer, which are often
much smaller, and the word \emph{stabilizer} is used to refer to the two
concepts interchangeably.

In fact, in the Gottesman-Knill Theorem, it was found that a class of quantum
operations could be simulated efficiently on a classical
computer by keeping track of the system dynamics with respect to the
stabilizer generators. This class corresponds to the Clifford group.

In 1997, Kitaev proposed a radically different kind of error-correcting code,
one based on the topology of logical qubits \cite{Fowler2009}.
By requiring logical operators to
use macroscopic movements of logical qubits around each other in a regular
lattice, this code can be made robust to local perturbations (noise). The
encoding can be easily scaled by increasing the lattice size to drive down the
the error probability.

%%%%%%%%%%
% FIGURE %
%%%%%%%%%%
%\begin{figure}
%\caption{Insert picture of surface code lattice here, with stars and plaquettes}
%\end{figure}

The surface code is closely related to the toric code but is appealing for
practical implementation because it can be done realistically
in 2D on a finite lattice, and it has a much higher threshold error rate
than concatenated error-correcting schemes (approaching 1%).
The surface code can also be expressed in the stabilizer formalism. Here there
are two types of stabilizer generators, the $X$ type, which are also called
star operators because they involve $X$ operators on the $4$ qubits around a 
vertex, and the $Z$ type, which are also called plaquette operators, which
involve $Z$ operators on the $4$ qubits around a square.

Just as with other stabilizer codes, it is not possible to do non-Clifford
gates fault-tolerantly in the surface code. Therefore, we must create
non-stabilizer states, inject them into our code, and perform controlled
rotations on them targeting the logical qubits on which we wish to enact
a non-Clifford gate. These non-stabilizer states that would allow us to
do $\textsc{UQC}$ are magical in this sense, and we will describe them in the next
section.

%Also, we distinguish here topological error codes, which use topological
%properties of encoded qubits which can be implemented on any physical
%technology, with physically topological qubits, which use quasiparticle
%excitations (anyons) directly to implement qubits. Cite a bunch of papers
%by Station Q, as well as the original proposal by Kitaev.

\subsection{Parallelism}

It is now the point to mention the close relationship between error correction,
parallelism, and classical control. Because errors can occur at any time,
and the coherence of quantum states are very fragile, error syndromes must be
measured frequently. The syndrome measurements are operations done by a
classical controller, and dictate the error correction procedure, which is
also classical. For large numbers of qubits, it will probably not be
feasible to detect and correct errors on qubits one at at ime. Rather, we would
prefer to detect and correct on all qubits in parallel. Therefore, at a
macroscopic level, error correction is inherently parallel and involves
classical control. For each qubit, physical or a logical at any
layer of concatenated encoding, we would also like the error detection
and correction procedure to be fast and efficient, but this is beyond the
scope of the current report.