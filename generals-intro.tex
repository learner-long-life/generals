\section{Introduction}
\label{sec:intro}

Quantum computation has emerged as the most general model of what is
physically possible in our universe, redefining a computing machine as
making use of the full power of our most realistic theory of physics
at a small scale, quantum mechanics. However, great engineering challenges
currently stand between us and a fully working quantum computer, mostly
related to fault-tolerance and scalability.

The original and most popular model of universal quantum computing is the
quantum circuit model, which is what we examine in this report.
Quantum circuits, like their classical counterpart, consist of gates
with outputs which can be fed into the inputs of other gates
to construct larger circuits, with the restriction that these gates be
reversible and unitary.

In this report, we study an emerging trend in quantum architecture
research, that of low-depth quantum circuits, and we use it to unite
the important themes of classical control, parallelism, fault-tolerance,
and circuit resources. Our approach is one of pragmatism with an
eye towards a new generation of quantum computer architects and engineers
working closely with both experimental physicsts and theorists to translate
algorithms onto running hardware. We use mathematical formalism where we
think it is illustrative, but we will not get stuck on rigorous proofs here.
Quantum computers are like the cathedrals of the 21st century, and their
architects may not live long enough to see their designs get built unless
we make these problems accessible to as many builders and thinkers as possible.

Here then is the organization of our report.
In Section \ref{sec:circuit}, we define the quantum circuit model and the
resources that are measured and minimized in quantum architecture.
In Section \ref{sec:parallel}, we survey some recent
parallelization techniques
behind the recent push towards constant-depth circuits.
In Section \ref{sec:qft}, we apply this circuit model to the development of
a core building block, the quantum Fourier transform, and follow its
various optimizations over the years. In Section \ref{sec:factor}, we
discuss the problem of factoring integers with a view towards approximation
and reducing depth.
In Section \ref{sec:ft}, we discuss the connection of parallel classical
control to fault-tolerance and error-correcting codes.
In Section \ref{sec:magic}, unify the ideas of uninitialized qubits and
error correction to the problem of
magic state distillation. In Section \ref{sec:hamsim}, we suggest
possibilities for parallelizing
the simulation of a quantum system.
Finally, we conclude in Section \ref{sec:conclude} with
some interesting directions for future research to explore the themes developed
in this report.
