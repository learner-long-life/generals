\section{Introduction}
\label{sec:intro}

Quantum computation has emerged as the most general model of what is
physically possible in our universe, redefining a computing machine as
making use of the full power of our most realistic theory of physics
at a small scale, quantum mechanics. However, great engineering challenges
currently stand between us and a fully working quantum computer, mostly
related to error correction, fault-tolerance, and maintaining delicate
control over microscopic systems while still interacting with them for
preparation and readout.

Some people have chosen to work on a more restrictive problem using
the quantum computing model of adiabatic optimization (cite D-Wave paper here).
The original and most popular model of universal quantum computing is the
quantum circuit model, which is what we examine in this report.
Quantum circuits, like their classical counterpart, consist of gates
whose inputs and outputs are connected to construct larger structures,
such that feeding in inputs in one end will give us our desired outputs at
the other end. In this case, however, the circuits are reversible.

Something about compiling also here, and references to
\cite{Pham2012a}. We define here quantum architecture as the physical
layout of qubits and their interactions in order to run quantum algorithms
while minimizing circuit resources \cite{VanMeter2006}, \cite{Pham2012b}.


In this report, we study an emerging trend in quantum circuit construction
research, that of low-depth quantum architectures, and we use it to unite
the important themes of classical control, parallelism, fault-tolerance,
and circuit resources. In doing so, we define several new resources to
augment the existing circuit model, namely \emph{circuit space}, its
combination with \emph{success probability}, and the use of
\emph{unitialized qubits}. Our approach is one of pragmatism with an
eye towards a new generation of quantum computer architects and engineers
working closely with both experimental physicsts and theorists to translate
algorithms onto running hardware.

Here then is the organization of our report.
In Section \ref{sec:circuit} we define the quantum circuit model and the
resources that are measured and minimized in quantum architecture. In
Section \ref{sec:qft} we apply this circuit model to the development of
a core building block, the quantum Fourier transform, and follow its
various optimizations over the years. In Section \ref{sec:factor} we
discuss the celebrated problem of factoring integers with a view to our new
model. In Section \ref{sec:parallel} we discuss some generic
parallelization techniques, due to Hoyer, Spalek, Takahashi, and Tani,
that have been behind the recent push towards constant-depth circuits.
In Section \ref{sec:ft} we discuss the connection of parallel classical
control and unitialized qubits to fault-tolerance and error-correcting codes.
In Section \ref{sec:magic} we find another unifying theme in the literature,
which is the source of a beautiful geometric perspective to UQC, called
magic state distillation. In Section \ref{sec:hamsim} we apply our model to
measure the performance of a useful problem in quantum computing,
namely that of simulating the Hamiltonian of a quantum system.
Finally, we conclude in Section \ref{sec:conclude} with some open problems
and interesting directions for future research to explore the themes developed
in this report.

%%%%%%%%%%%%%%%%%%%%%%%%%%%%%%%%%%%%%%%%%%%%%%%%%%%%%%%%%%%%%%%%%%%%%%%%%%%%%%
\subsection{The Rules of Quantum Mechanics}

An extensive background in quantum physics is not necessary to begin
working in and appreciating quantum computing or the work in this report.
The interested reader is referred to the standard textbook in this field
\cite{Nielsen2000}, but we will cover the mathematical basics here.
In general, quantum mechanics for the purposes of computing can be
explained in four rules.

Here I want to use Latex definitions with boldface for the terms, but I
forgot how to do that.

%%%%%%%%%%%%%%%%%%%%%%%%%%%%%%%%%%%%%%%%%%%%%%%%%%%%%%%%%%%%%%%%%%%%%%%%%%%%%%
\subsection{Rule 1: Qubits}

\begin{definition}
Rule 1 [Qubits]: Our basic unit of information is a quantum bit (\emph{qubit}),
a two-level system which can be represented by a column vector with
complex coefficients whose magnitudes squared sum to 1. These are known
as probability amplitudes for reasons which will become clear in Rule 4.
A quantum state
is written using Dirac bra-ket notation due to a visual pun for the
inner product, or the overlap between two quantum states.
The states $\ket{0}$ and $\ket{1}$ are known as the computational basis,
and correspond to our normal idea of classical bits.
\end{definition}

\begin{equation}
\ket{\psi} = \alpha \ket{0} + \beta \ket{1} \qquad
|\alpha|^2 + |\beta|^2 = 1
\end{equation}

There is a nice correspondence between a single-qubit state and a geometric
picture known as the Bloch sphere, shown in Figure \ref{fig:bloch-sphere}.
We will return to this Bloch sphere picure later in talking about
magic state distillation in Section \ref{sec:magic}.
The north pole represents $\ket{0}$ and the south pole represents $\ket{1}$.

\begin{figure}
\label{fig:bloch-sphere}
\caption{The famous Bloch sphere}
\end{figure}

%%%%%%%%%%%%%%%%%%%%%%%%%%%%%%%%%%%%%%%%%%%%%%%%%%%%%%%%%%%%%%%%%%%%%%%%%%%%%%
\subsection{Rule 2: Tensor Product}

Individual qubits are not very useful, so we
can combine via the tensor product to form larger quantum
states. A two-qubit quantum state can exist as a complex superposition of
four two-bit classical states, still with probability amplitudes that sum to
one.

\begin{equation}
\ket{\psi} = \alpha \ket{00} + \beta \ket{01} + \gamma \ket{10} + \delta \ket{11}
\end{equation}

%%%%%%%%%%%%%%%%%%%%%%%%%%%%%%%%%%%%%%%%%%%%%%%%%%%%%%%%%%%%%%%%%%%%%%%%%%%%%%
\subsection{Rule 3: Unitary Operations}

\begin{definition}
Rule 3 [Unitary Operations]:
Qubits can be transformed by operations (also called \emph{gates})
that take the form of $2\times 2$ unitary matrices of unit determinant.
That is, they are reversible ($UU^\dagger = I$) and preserve the magnitude
of quantum states.
\end{definition}

\begin{equation}
\left[ \begin{array}{cc}
a & b \\
c & d \\
\end{array} \right]
\end{equation}

%%%%%%%%%%%%%%%%%%%%%%%%%%%%%%%%%%%%%%%%%%%%%%%%%%%%%%%%%%%%%%%%%%%%%%%%%%%%%%
\subsection{Rule 4: Measurement}

\begin{definition}
Rule 3 [Measurement]: While a qubit can exist in a superposition
of computational basis states, when measured (that is, interacted with a
large macroscopic object like a human-sized scientific instrument), it
collapses to either $\ket{0}$ with probability $|\alpha|^2$ or
$\ket{1}$ with probability $|\beta|^2$.
\end{definition}

