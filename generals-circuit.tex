\section{Quantum Circuit Resources}
\label{sec:circuit}

A \emph{quantum architecture} is a physical layout of qubits which constrains
which qubits can interact with each other, usually optimized to run a particular
``hard-coded'' algorithm. Like classical architecture, it is the
connecting layer between mathematical formalism and experimental implementation.
There are two kinds of overhead studied in quantum architecture,
corresponding to compiling, which we will discuss in \ref{subsec:compile},
and error-correction, which we defer until Section \ref{sec:ft}. These
overheads, and their improvements, are usually of the polynomial variety.
Nevertheless, these may be numerically significant enough so that your
algorithms terminate before you lose dramatic tension with
your funding agency.

%In the rest of this section, we will describe the quantum circuit model,
%give an overview of different models for
%quantum architectures, discuss universal gate sets,
%and then describe quantum compilation in more detail.

%%%%%%%%%%%%%%%%%%%%%%%%%%%%%%%%%%%%%%%%%%%%%%%%%%%%%%%%%%%%%%%%%%%%%%%%%%%%%%
\subsection{The Quantum Circuit Model}

Quantum architecture and their resources are described on top of the framework of
the quantum circuit model, where gates (represented by blocks)
are dynamically applied over time
from left to right 
to stationary qubits (represented by horizontal lines) which usually begin in some
fiducial state $\ket{0}^{\otimes n}$ and end in a classical, projective measurement
to read out a result.
%although classically-controlled measurements may also
%occur in the ``middle'' of the circuit.
%This definition abstracts away many interesting physical
%techniques, such as the shuttling of trapped ions around a surface with
%tracks to change their interactions \cite{Blakestad2011},
%but it will suffice for our discussion.
%In the quantum circuit model, qubits are represented by lines,
%and gates here appear as blocks that operate on one or two qubits in a single
%time-step, as time flows from left to right. These qubits are not necessarily
%related to each other spatially.

\begin{figure}
\begin{center}
\begin{displaymath}
W = 5 \left\{
\begin{array}{c}
S = 15 \\
\Qcircuit @C=1em @R=.7em { 
	& \gate{H} & \qw & \ctrl{1} & \gate{H} & \qw & \qw      & \ctrl{1} & \qw \\ 
	& \gate{H} & \qw & \targfix & \ctrl{2} & \qw & \gate{K} & \targfix & \qw \\
	& \gate{H} & \qw & \gate{K} & \qw      & \qw & \gate{H} & \qw      & \qw \\
	& \gate{H} & \qw & \ctrl{1} & \targfix & \qw & \gate{H} & \qw      & \qw \\
	& \gate{H} & \qw & \targfix & \gate{H} & \qw & \qw      & \qw      & \qw
	\gategroup{1}{2}{5}{9}{.7em}{--}
}\\
\xymatrix {
  & & D=5 \ar[ll] \ar[rr] & & \\
 }
\end{array}
\right.
\end{displaymath}
\caption{Circuit size $S$, depth
$D$, and width $W$}
\label{fig:circuit-resource}
\end{center}
\end{figure}

Three circuit resources are commonly measured in literature to judge the
quality of an architecture in implementing an algorithm, as shown in Figure
\ref{fig:circuit-resource}.
\emph{Circuit size} is the number of non-identity gates applied over time, and
is roughly proportional to the square area that a circuit diagram takes up
on paper. In
reality, a qubit may have to undergo active error-correction to even maintain
a coherent state, but identity gates are usually disregarded at the algorithmic
level.
\emph{Circuit width} is the total number of qubits used in a computation,
including those storing the input and the output, and visually corresponds to
the number of lines or the height on paper of the circuit diagrams.
\emph{Circuit depth} describes the number of parallel timesteps needed to
complete a computation, intuitively the horizontal length on paper of a circuit,
where each gate takes a unit timestep. It is the delay
we can expect from entering our inputs until we get a result, and it takes
center stage in this report. It is the one that all the cool kids try to
optimize. However, all three circuit resources interact
with each other---decreasing depth usually increases size and width analogous
to a classical time-space tradeoff.

%%%%%%%%%%%%%%%%%%%%%%%%%%%%%%%%%%%%%%%%%%%%%%%%%%%%%%%%%%%%%%%%%%%%%%%%%%%%%%
\subsection{Circuit Space}

We now introduce a new circuit resource called the \emph{circuit space},
denoted by $Q$,
equal to the size of the coherent state over the lifetime of an algorithm.
This is inspired by the generic entangled resource state created initially
in the measurement-based quantum computing (MBQC) model
model of Raussendorf, Browne, and Briegel \cite{Raussendorf2003},
which also inspired the unbounded fan-out gate behind many of the constant-depth
circuits mentioned in this report.

The intuition behind defining circuit space is that we only need to
error-correct states while they are entangled with the output register.
While input qubits to the circuit generally begin in a classical
product state, as the circuit executes, eventually all qubits become entangled
with each other in a single computation output state
in order to affect the output, with the exception of
ancillae qubit which may become uncomputed back to $\ket{0}$ and therefore
leave the computation state again.

We can trace back this computation state from the output by reversing the
history of the circuit and see how it
branches as we return to the inputs, and include it in our cumulative
circuit space.
Decreasing circuit depth reduces the amount of time that the circuit space
must be kept coherent, but it may also increase the circuit
width over time and therefore increase the circuit space itself.

The circuit space is upper bounded by the product of the circuit width and
circuit depth, since in the worst case every qubit is being operated on with
an entangling operation for the entire duration of the circuit.
The circuit space is lower bounded by the circuit size, since we cannot
increase the computation state with an entangling gate without also
adding to the circuit size with that gate. Between those two bounds, there is
a lot of room for variability and improvement.

\begin{equation}
S \le Q \le D\cdot W
\end{equation}

%This is not an exact definition, but it gives a flavor of the kinds of
%circuit resources which are worth optimizing in the tradeoff in between
%size, width, and depth, now that many circuits have saturated the
%constant-depth lower bound.

%%%%%%%%%%%%%%%%%%%%%%%%%%%%%%%%%%%%%%%%%%%%%%%%%%%%%%%%%%%%%%%%%%%%%%%%%%%%%%
\subsection{Architectural Models}

We follow here the terminology of Van Meter \cite{VanMeter2006} in his
definition of architectural models. Most algorithm designers assume
that arbitrary qubits can interact in a quantum circuit.
This is the simplest, but least
physically realistic model, called the Abstract Concurrent ($\textsc{AC}$) model.
Single- and two-qubit gates are allowed to operate within the same
timestep on disjoint qubits.

However, in the laboratory, it is usually the case that only spatially-local
qubits can interact with each other. If we restrict the physical layout of
the qubits to be in one dimension, such as in a linear ion trap, and
allow only nearest-neighbor two-qubit gates (with the same concurrency and
single-qubit gates as $\textsc{AC}$, then we get the model called
$\textsc{1D NTC}$. If we allow the qubits to be laid out in two dimensions,
such as in modern surface ion traps \cite{Blakestad2011} or
proposed topological qubits using
Majorana fermions on 1D nanowires \cite{Bonderson2010}, we get the
model called $\textsc{2D NTC}$. Intuitively, this extra degree of freedom allows
data to move around each other efficiently using constant-depth teleportation.
This has been shown to reduce the depth of arithmetic and factoring
algorithms running on 2D architectures over
their 1D counterparts \cite{Choi2010} \cite{Pham2012b}, from $O(n)$ to
$O(\sqrt{n})$ or even $O(\log^2 n)$.
Naturally, this can be extended to higher dimensions, as shown in the
context of unbounded fan-in controlled operations in $k$ dimensions for
Grover's search algorithm \cite{Rosenbaum2012}. These architectures are
called $k\textsc{D NTC}$, and without classical control they have a 
lower bound of 
$O(\sqrt[k]{n})$ depth.

%%%%%%%%%%%%%%%%%%%%%%%%%%%%%%%%%%%%%%%%%%%%%%%%%%%%%%%%%%%%%%%%%%%%%%%%%%%%%%
\subsection{Universal Gate Set}
\label{subsec:universal}

In classical computing, the two-input NAND gate is universal for producing
all boolean functions, and indeed, all classical processors consist of
transistors which essentially perform NAND and similar logical operations
efficiently. What is the equivalent gateset for performing universal
quantum computation (\textsc{UQC})?

Due to the requirements of error-correction, we can only perform a discrete
set of
quantum gates fault-tolerantly. In many error correcting codes, this gate
set corresponds to the \emph{Clifford group}, which we denote $\mathcal{C}_n$.
This group is defined as the
normalizer of the Pauli group $\mathcal{P}_n$.
That is, it is the set of gates which
when conjugated around a Pauli operator still stays within $\mathcal{P}_n$.

For single-qubit operations, the single-qubit Clifford group $\mathcal{C}_1$
is generated by the elements $H$ (called the Hadamard operator),
and $K$ (called the phase gate) along with arbitrary phases as defined below:

\begin{equation}
H = \normtwo
 \left[
  \begin{array}{cc}
    1 & 1 \\
    1 & -1 \\
  \end{array} \right]
\qquad
K = 
 \left[
  \begin{array}{cc}
    1 & 0 \\
    0 & i \\
  \end{array} \right]
\end{equation}

There are 24 elements in $\mathcal{C}_1$, which also include the Pauli $X$ and
$Z$ matrices.

In the two-qubit Clifford group $\mathcal{C}_2$, our single-qubit
generators are the
same as for $\mathcal{C}_1$, on either the first or second qubit,
and we add the CNOT (controlled NOT) gate, sometimes written as $\Lambda(X)$, which
is the same as a classical reversible XOR operation.

\begin{equation}
\Lambda(X) = 
 \left[
  \begin{array}{cccc}
    1 & 0 & 0 & 0\\
    0 & 1 & 0 & 0\\
    0 & 0 & 0 & 1\\
    0 & 0 & 1 & 0
  \end{array} \right]
\end{equation}

However, even that is not sufficient to be universal. We must add either
the Toffoli gate, which is a doubly-controlled NOT sometimes written as
$\Lambda^2(X)$, or the $R_Z(\pi/8)$ gate.
%, sometimes called the $\pi/8$ gate
%due to the double-covering between $SU(2)$ and $SO(3)$.

\begin{equation}
\Lambda^2(X) = 
 \left[
  \begin{array}{cccccccc}
    1 & 0 & 0 & 0 & 0 & 0 & 0 & 0\\
    0 & 1 & 0 & 0 & 0 & 0 & 0 & 0\\
    0 & 0 & 1 & 0 & 0 & 0 & 0 & 0\\
    0 & 0 & 0 & 1 & 0 & 0 & 0 & 0\\
    0 & 0 & 0 & 0 & 1 & 0 & 0 & 0\\
    0 & 0 & 0 & 0 & 0 & 1 & 0 & 0\\
    0 & 0 & 0 & 0 & 0 & 0 & 0 & 1\\
    0 & 0 & 0 & 0 & 0 & 0 & 1 & 0
  \end{array} \right]
\end{equation}
\begin{equation}
 R_Z(\pi/8) = 
 \left[
  \begin{array}{cc}
    1 & 0 \\
    0 & e^{i\pi/4} \\
  \end{array} \right]
\end{equation}

The Toffoli gate itself can be decomposed into single- and two-qubit
operations involving gates from $\mathcal{C}_2$ and $R_Z(\pi/8)$.
Both the Toffoli gate and $R_Z(\pi/8)$ are harder to implement
fault-tolerantly than
Clifford group elements, and minimizing their occurrence in quantum
algorithms is a mostly unexplored area in quantum architecture research.

For the purposes of compiling in this report, we can define our universal gate set
as follows.

\begin{equation}
\mathcal{G} = \mathcal{C}_2 \cup \{ R_Z(\pi/8) \}
\end{equation}

%%%%%%%%%%%%%%%%%%%%%%%%%%%%%%%%%%%%%%%%%%%%%%%%%%%%%%%%%%%%%%%%%%%%%%%%%%%%%%
\subsection{Quantum Compiling}
\label{subsec:compile}

Quantum compiling, like
classical compiling, translates arbitrarily complex high-level ``software''
operations to physically simple,
universal, machine-dependent ``assembly'' instructions.

\begin{figure}
\begin{center}
\begin{displaymath}
\begin{array}{ccc}

%%%%%%%%%%%%%%%%
% Source circuit
%\underbrace{
\begin{array}{c}
S = 2 \\
\Qcircuit @C=0.5em @R=.5em { 
	& \multigate{4}{U_1} & \qw & \multigate{4}{U_2} & \qw \\ 
	& \ghost{U_1}        & \qw & \ghost{U_2}        & \qw \\
	& \ghost{U_1}        & \qw & \ghost{U_2}        & \qw \\
	& \ghost{U_1}        & \qw & \ghost{U_2}        & \qw \\
	& \ghost{U_1}        & \qw & \ghost{U_2}        & \qw 
	\gategroup{1}{2}{5}{4}{.7em}{--}
}\\
\xymatrix {
  & D=2 \ar[l] \ar[r] & \\
 }
\end{array}
%}_{C}

%& 
%\begin{array}{c}
%\textsc{Quantum Compiler} \\
\rightarrow
%\end{array}
%&

%%%%%%%%%%%%%%%%
% Target circuit
%\underbrace{
\begin{array}{c}
S' = 15 \\
\Qcircuit @C=0.5em @R=.5em { 
	& \gate{H} & \qw & \ctrl{1} & \gate{H} & \qw & \qw      & \ctrl{1} & \qw \\ 
	& \gate{H} & \qw & \targfix & \ctrl{2} & \qw & \gate{K} & \targfix & \qw \\
	& \gate{H} & \qw & \gate{K} & \qw      & \qw & \gate{H} & \qw      & \qw \\
	& \gate{H} & \qw & \ctrl{1} & \targfix & \qw & \gate{H} & \qw      & \qw \\
	& \gate{H} & \qw & \targfix & \gate{H} & \qw & \qw      & \qw      & \qw
	\gategroup{1}{2}{5}{9}{.7em}{--}
}\\
\xymatrix {
  & & D'=5 \ar[ll] \ar[rr] & & \\
 }
\end{array}
%}_{C}

\end{array}
\end{displaymath}

\caption{A quantum compiler in action}
\label{fig:qcompile}
\end{center}
\end{figure}

Quantum compilers take a
generic unitary matrix $U$ on $n$-qubits, that is, an element of $SU(2^n)$, and
attempts to approximate it with a sequence of gates
(that is, a product of some matrices $V_1\cdots V_k$ )
from $\mathcal{G}$
defined in the previous section, using some distance metric like
operator norm.

\begin{equation}
|| U - V_1\cdots V_k || < \epsilon
\end{equation}

Quantum compilers are generally classical algorithms that run on your
digital computer and produce a deterministic sequence of quantum gates, which
you can then run in your laboratory (for example, by controlling laser pulses
for trapped ions). They accept an input circuit and produce a compiled
output circuit that approximates it, with some increase in circuit size,
width, and depth.

Compilation is therefore
one area where one can generically decrease circuit depth.
%, independently of the
%algorithm.
The first efficient quantum compiler due to
Solovay and Kitaev (SK) had a depth overhead of $O(\log^3+\nu(n/\epsilon))$,
where $\nu$ is a constant which can be made arbitrarily small, and no
ancillae (the width was $O(n)$) \cite{Dawson2005}.
A more parallel compiling procedure was discovered by Kitaev, Shen, and
Vyalyi (KSV) with a depth overhead of $O(\log n + \log\log(1/\epsilon))$ but
with a width of $O(n^3)$ on an $\textsc{AC}$ architecture \cite{Kitaev2002}.
Numerical comparisons between SK and KSV can be found in Ref. \cite{Pham2012a}.
Moreover, the KSV procedure relies on a prior decomposition of
a unitary matrix in $SU(N=2^n)$ into (controlled) single-qubit gates,
for example the multiply-controlled single-qubit gates in
the palindrome transform of Aho and Svore \cite{Aho2003}.

Austin Fowler's compiler uses a brute force search over group elements of
$\mathcal{C}_1$ alternating with $R_Z(\pi/8)$ which still runs in time
exponential in sequence length.
This may be optimized using classical parallelization and hardware acceleration
\cite{Booth2012}, possibly also for scaling this approach to
two-qubit compilation with $\mathcal{C}_2$ (containing 11,520
elements).