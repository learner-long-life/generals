\section{Quantum Circuit Resources}

Here we can mostly copy from my other papers, where I define quantum
architecture as being the connecting layer between quantum algorithms
and physical experiment. It's a really necessary connection, since otherwise
nice algorithms are ignored or unknown by experimentalists for lack of a
realistic mapping or being the resources are still too huge. Therefore, the
gains to be made in quantum architecture are usually not of the exponential
variety, but are polynomially gains asymptotically, which may turn out to be
huge in practice with actual constants. This may make the difference between
implementing a quantum computer in a few years versus several decades, if you
think we can sustain dramatic tension with DARPA / NSA / IARPA that long.

\subsection{The View}

I also mention here the field of quantum computer engineering, which will
include as its subfields quantum compilers (gate construction) and quantum
architecture (layout and error correction). I am mostly concerned

The tone of most of my paper is therefore more practically minded. Sometimes
mathematical formalism is necessary to realize substantial savings, but I will
not be overly concerned with nicely erasing my ancillae for the next algorithm
in line, especially if it means a large number of operations to uncompute it
to save a relatively small number of ancillae.

Also, I will stick to asymptotic improvements where possible, since these
statements are less likely to change over time or be dependent on
as-yet-unknown physical implementation. However, as an engineer I am still
impressed by numerical improvements, and on occasion I will use actual
constants to call out changes which would otherwise be lost in $O(\cdot)$
notation. This may be more appropriate in an intro section, which also
includes the rules of quantum computing.

\subsection{Circuit Resources}

In the traditional quantum circuit model, qubits are represented by lines,
and operations over the flow of time going from left to right. Of course,
due to the way matrix multiplication works, in real-life implementations,
the gates are applied going from right to left. Gates are represented by
blocks, and can be single-qubit (occurring on a single line) or
multi-qubit (spanning multiple lines).

\begin{figure}
\label{fig:circuit-resource.tex}
\caption{The circuit resources of width, depth, and size as described by others.}
\end{figure}

\emph{Circuit size} is the number of non-identity gates applied over time.
We specify non-trivial here, since in reality the identity gate may involve
extensive error-correction just to maintain a coherent quantum state over
a timestep, even without enacting any logical operation on it.
\emph{Circuit width} is the total number of qubits used in a computation,
including those storing the input and the output. It is fashionable in other
papers to only count ancillary qubits (ancillae), that is, the temporary 
scratch space used by the computation, which start out in the $\ket{0}$ state
and must be returned to it. This makes the circuit width smaller, and a more
impressive figure. However, this makes it hard to compare circuits
which compute the output in-place, ``on-top'' of the input qubits to save
space, to speak, versus those who compute out-of-place. Furthermore,
the output of one circuit is often the input to another, and in a practical sense,
we are interested in how many qubits we must physically maintain over time in
our experiment, and not whether these qubits store the inputs or outputs per se.
\emph{Circuit depth} is in some ways the most important figure. It describes
the number of parallel timesteps needed to complete a computation, given enough
parallel classical controllers. It provides a lower bound on the running time,
and like classical circuit depth, is analogous to the delay we can expect
from entering out inputs to getting a result.

Quantum architecture is primarily concerned with reducing circuit resources,
and previous approaches have concentrated on reducing circuit depth, circuit
width, or some combination of both (cite here the LNN factoring papers,
Van Meter's stuff).

\subsection{Architectural Models}

We follow here the terminology of Van Meter \cite{VanMeter2006} in his
definition of architectural models with different constraints, going from
very abstract but simple to more realistic but with more complicated constraints.

The qubits in the circuit models above are not
necessarily related spatially, and arbitrary interactions are allowed.
This is called the Abstract Concurrent (\textsc{AC}) model.

If we stipulate that the qubits then must be spatially related, in that
neighboring qubits in the circuit diagram are actually neighboring in real-life,
we take the natural shape of the entire arrangement of qubits to be in a line.
For the sake of realism, we only allow neighboring-qubits to interact, and we
only allow single-qubit and two-qubit gates. Due to various universality
proofs and fault-tolerance, it suffices to restrict ourselves to single-qubit
operations which are from the Clifford group and the $\pi/8$ gate and CNOT.
(Cite here Nielsen & Chuang, or KSV). We allow concurrent operations
acting on disjoint sets of qubits to occur within the same timestep.
This is called \textsc{1D-NTC}.

Next, it becomes natural to extend this model into two dimensions,