\section{Magic State Distillation}

Now we come to a beautiful idea which combines a geometric picture of
quantum computing with error correction, circuit construction, the idea of
using uninitialized quantum states, and architecture.
This idea is the \emph{magic state}.

In 1998, Bravyi and Kitaev proposed an interesting model of computation where
we assume we are able to perform the following set of operations $O$
perfectly: all the Clifford group operations,
prepare fiducial state $\ket{0}$, perform Pauli measurements, but in
addition we are given multiple copies of some mixed state $\rho$. This last
is a parameter of our model. Under what conditions of $\rho$ are we still
able to do universal quantum computation (\textsc{UQC})? It turns out that
if $\rho$ is a pure state corresponding to the eigenstates of the Hadamard
($H$) operator or the $T$ operator, as defined below, then combined with the
above operations we are able to perform \textsc{UQC}. Note that the
$T$ operator here is \emph{not} the same as the $\pi/8$ gate, or $R_Z{\pi/8}$ that is
often called $T$ in the literature, for example in \cite{Nielsen2000}.
We will always refer to this as a $\pi/8$ gate to avoid confusion.
This is an unfortunate conflation, since the $\pi/8$ gate, together with the
Clifford group, is necessary for \textsc{UQC}.

We actually want the two equations below to be displayed side-by-side.
These actually need heavy fact-checking, I am just bullshitting from
memory.

\begin{displaymath}
H = \frac{1}{\sqrt{2}} \begin{array}{cc}
1 & 1 \\
1 & -1 \\
\end{array} = \frac{1}{2}(I + \sigma_x + \sigma_z)
\end{displaymath}

\begin{displaymath}
T = ???
\end{displaymath}

We denote by $\ket{H}$ and $\ket{T}$ the $+1$ eigenstates of $H$ and $T$
respectively. By the Clifford group operations, there are eight symmetric
magic states to $\ket{H}$ and twelve symmetric magic states to $\ket{T}$,
or $20$ magic states total. We will therefore distinguish between
magic states in the $\ket{H}$ direction, which can be identified as vectors
which go from the origin of the Bloch sphere to bisect the edges of the
Clifford octahedron, and magic states in the $\ket{T}$ direction, which
go through the centers of the faces of the Clifford octahedron.

I want to use the fancy $\theta$ in the equations below.

\begin{displaymath}
\ket{T} = \cos(\theta) \ket{0} + \sin(\theta) \ket{1} \qquad \theta = \frac{2}{3}???
\end{displaymath}

\begin{displaymath}
\ket{H} = \ket{0} + e^{i\pi/4} \ket{1}
\end{displaymath}

The use of the word ``magic state'' in the error-correcting literature,
particularly for topological quantum computing, often refers to $\ket{H}$,
because it is fairly easy to enact the $\pi/8$ gate fault-tolerantly using
such a magic state, for most error-correcting codes which are able to do
the Clifford gates fault-tolerantly (and transversally).
A circuit for doing this in the Steane code is shown in Figure \ref{fig:steane-pi8}.

\begin{figure}
\label{fig:steane-pi8}
\caption{Fault-tolerant implementation of the $\pi/8$ gate using an $\ket{H}$-type
magic state in the Steane code. \cite{Fowler2003}.}
\end{figure}

\subsection{Bloch Sphere}

We now return to the picture of the Bloch sphere from a previous section.
Whereas before we only considered pure states, those whose preparation could
be described in a deterministic way, this is not the most general definition
of a quantum state. In general, we may not know how a state was prepared,
either because a random error has occurred, or an adversary has tampered
maliciously with it, or because the original experimentalist who prepared
for the state for us is now unfortunately dead. This more general state,
called a \emph{mixed state}, can be expressed as a probabilistic ensemble of
states $\rho$, which is now a $2^n \times 2^n$ density matrix. All the
rules of quantum mechanics from the introduction can be reformulated in
terms of density matrices. For this report,
we will limit ourselves to single-qubit states with the understanding that
the framework generalizes naturally to multi-qubit states.

Although we
will not delve deeply into the properties of density matrices here, they are
a tool especially well-suited to expressing quantum states which are part
of a larger system which may be unknown. Here is where philosophers or
physicists who work in the foundations of quantum mechanics argue about whether
quantum states represent objective states of nature or simply our beliefs about
these states and the imperfect information we have about them. We will leave
that debate to them.

\begin{displaymath}
\rho = \sum_{j} p_j \ket{\psi_j}\bra{\psi_j}
\end{displaymath}

Here the probabilities $p_j$ should sum to $1$ to be a true probability
distribution. This is also how a mixed quantum state can be viewed as a
generalization of a classical probability distribution. In fact, many
quantum algorithms involve the ability to sample from a ``hard'' distribution,
that is, one that cannot even be sampled from efficiently using a classical
probabilistic algorithm, let alone generate the complete distribution itself.
(Citation needed here, and also fact-checking. For example, sampling from
the Gibbs distribution may not involved mixed states at all, like the
quantum-quantum Metropolis algorithm).

In general, a mixed state can be expressed as a combination of the Pauli
matrices.

\begin{displaymath}
\rho = \frac{1}{2}(I + p_x\sigma_x + p_y\sigma_y + p_z\sigma_z)
\end{displaymath}

The values $(p_x, p_y, p_z)$ is known as the \emph{polarization vector} of
the mixed state and corresponds to its geometric three-dimensional coordinates
within the Bloch sphere (now properly considered a Bloch ball, since mixed
states can occupy the interior of the ball's volume). The real numbers
$p_x$, $p_y$, and $p_z$ lie in the range $[0,1]$.

Mixed states include the pure states as a special case, that is, all the
points that lie on the Bloch sphere surface. To restrict mixed states
to those points which lie within the Bloch ball, including the surface,
we impose $|p_x| + |p_y| + |p_z| \le 1$. We can also describe the
correspondence between the polarization vector and our conventional
description of a quantum state as follows (insert here the problem from
Aram's first homework about exponentiating the dot product of the polarization
vector with the Pauli matrices).

It turns out, by using the stabilizer framework (note here that stabilizer
framework should come before this section) and the Gottesman-Knill theorem,
we can classically simulate all the states within the interior of an
\emph{octahedron}, which we will call $\mathcal{O}$, which is defined as the
convex hull of the six points corresponding to the eigenstates of the
$X$, $Y$, and $Z$ operators.
This is shown in Figure \ref{fig:clifford-octahedron}.
These therefore known as stabilizer states, and we will sometimes refer to
$\mathcal{O}$ as the Clifford octahedron because all of these states can be
reached using the operations in $O$, which include the $\mathcal{C}_1$.
Note to self, this section should come after the section explaining the
Clifford group. It is therefore not surprising, although still somewhat
mysterious, that the 24 elements of the $\mathcal{C}_1$ correspond
exactly to the symmetries of the octahedron. (I think this is $S_4$, this
could use some fact-checking love).

\begin{figure}
\label{fig:clifford-octahedron}
\caption{The Clifford octahedron}
\end{figure}

\subsection{Error Decoding}

Now here comes the connection to error decoding. We can view every mixed
single-qubit state $\rho$ as the result of randomly applying any unitary
operator (let's choose $H$ for now) with probability $p$.
But does that mean it is a mixture of
the $+1$ and $-1$ eigenstates, when the unitary is applied and when it is not?
That doesn't make sense to me. But we'll go with it for now.
We will denote by $\ket{H^+}$ the $+1$ eigenstate of $H$ and by
$\ket{H^-}$ the corresponding $-1$ eigenstate. Actually, I am also still not
clear how the $+1$ and $-1$ eigenstates relate to each other geometrically
on the Bloch sphere.

The below needs to be fact-checked.

\begin{equation}
\rho = p\ket{H^+}\bra{H^+} + (1-p)\ket{H^-}\bra{H^-}
\end{equation}

That is, we can describe \emph} mixed state $\rho$ as a noisy preparation
of the eigenstate $\ket{H^+}$ with some error probability $p$, and we already
know how to correct errors to an arbitrarily low-precision: using
concatenated error correction! In this case, we consider our $n$ copies of
$\rho$ to have been ideally prepared as $\ket{H^}}$, but that noise has
perturbed it. I don't know if this picture is exactly correct, since Aram
seemed surprised by this, so I will need to think about this some more.
So we then apply concatenated levels of error decoding to
recover the initial state. In each layer of concatenated decoding, we get
out a qubit that is closer to $\ket{H^+}$ than its inputs. This is where
the ``distillation'' name comes from in this procedure, often called
\emph{magic state distillation}.
However, just as in conventional concatenated
error correction, there is a threshold for $p$ beyond which we cannot
recover.

More fact-checking needed below for this equation to describe how
a layer of Steane decoding reduces $p$ and distills a more ``magic'' state.

\begin{equation}
p_out = \frac{something p}{something else p} ????
\end{equation}

This threshold then divides up the Bloch ball into regions which can eventually
be distilled to a magic state, either in the $\ket{H}$-direction or the
$\ket{T}$-direction, or states which are classically-simulatable. Definitely
we know that states within the Clifford octahedron $\mathcal{O}$ belong to this
latter category, and that the points on the sphere corresponding to
$\ket{H}$ and $\ket{T}$ have unquestionable magicness. (What is the word meaning
``a magical quality''?) What do we know about pure states, on the surface
of the sphere? (Seriously, what do we know? This is not a rhetorical question.
Fact check!) However, that leaves the region outside the octahedron and within
the surface of the sphere unaccounted for, and this can be a fairly big space...
something something.

We know that there is a tight separation in the $\het{H}$ direction right up
to the edge of $\mathcal{O}$. That is, points in the plane that intersect
the edges of $\mathcal{O}$ and the origin, that is, bisecting both the
octahedron and the Bloch ball, but outside of $\mathcal{O}$ itself, can always
be distilled to $\ket{H}$. \cite{Bravyi2002}.

However, there is not a tight-separation in the $\ket{T}$ direction. What we
know is that points that lie beyond a plane $0.991$ (fact-check this!) from
the faces of $\mathcal{O}$ can be distilled to a magic state $\ket{T}$.
However, there is still an ambiguous region in between this plane and the
face of $\mathcal{O}$ $0.875???$ from the origin where it is not known
whether distillation is possible. This is an interesting open question.

\subsection{Practical Application / Uninitialized Qubits}

That's great, you may be thinking, but magic is fake. It certainly isn't a good
name if I want to build actual working quantum computers. Well, it turns out
that magic state distillation is an important part of several physical approaches
to building a quantum computer. This importance comes from the fact that the
$\pi/8$ gate is not efficient to implement in a fault-tolerant way directly
in either topological codes (including the Kitaev surface code), many CSS
codes used for concatenated error correction, and even in physically
topological systems such as those using the fractional quantum Hall effect.
(cite Station Q papers here).

What we can do is, using a (possibly large) number of uninitialized qubits
in some mixed state, distill them into a magic state $\ket{H}$ as an ancillary
resource to our normal circuit. Whenever we need to apply a $\pi/8$ gate to
a qubit, we make it the target of a CNOT controlled on our distilled magic
state $\ket{H}$. Therefore, there is a big area of quantum architecture which
becomes then, how to design around this large areas where magic state
distillation needs to happen, and where magic states need to be inserted
into the system. This depends on practical error rates and the number of
distillation layers needed, as determined below.

There is also a lot of work experimentally to be done. It could be that the
thresholds above in the previous section are nonsense, because empirically,
it is possible or easy to distill magic states. This can only be done
by trial and error (fact-check? maybe think about this statement some more?)

\subsection{A Quantitative Theory of Magic}

Wim van Dam has proposed an interesting idea about the interconversion of
magic states to non-magic states. In the above sections, we assumed that we
had $n$ mixed states $\rho$ and that if $n$ were large enough, we could
distill out a single magic state $\ket{H}$.